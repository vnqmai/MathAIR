\documentclass[12pt]{article}
\usepackage{fullpage,graphicx,psfrag,amsmath,amsfonts,verbatim}
\usepackage[small,bf]{caption}
\usepackage[utf8]{vietnam}

\input defs.tex

\bibliographystyle{alpha}

\title{[MathAIR'19.01] Bài tập chương 1}
\author{Võ Ngọc Quỳnh Mai}

\begin{document}
\maketitle

\newpage


\section{Trí tuệ là gì?}

Là khả năng học và áp dụng kiến thức, kỹ năng. Trí tuệ có nhiều loại hình như tự nhiên, âm nhạc, logic toán học, vận động, ngôn ngữ, nội tâm. Ngoài ra, trong lĩnh vực học máy của trí tuệ nhân tạo, trí tuệ còn thể hiện qua việc suy diễn từ kết quả ra nguyên nhân (causal reasoning), lập kế hoạch (planning), sáng tạo các giải pháp mới mẻ cho bài toán học máy,...

\section{Trí tuệ nhân tạo là gì? Những tiến bộ nào góp phần dẫn đến cuộc cách mạng AI?}
Trí tuệ nhân tạo là máy tính với một số chức năng vê trí tuệ giống của con người hay máy tính với khả năng học và áp dụng kiến thức, kỹ năng để giải quyết các bài toán thực tế trong đời sống.
Sự bùng nổ phát triển trong nhiều lĩnh vực đã góp phần dẫn đến cuộc cách mạng AI. Trong đó phải kể đến việc phân tích và khai thác nguồn dữ liệu khổng lồ (big data), nó được ứng dụng để tăng cường trí tuệ của con người trong hầu hết các lĩnh vực của đời sống như công nghệ AI/bots, thị giác máy tính, robot, an ninh mạng, lĩnh vực kinh doanh, marketing, y tế,...


\section{Học là gì? Học máy (machine learning; ML) là gì? Vị trí của ML trong AI? Kiến thức kỹ năng có thể được biểu diễn trong máy tính ra sao?}
Học là quá trình thu thập kiến thức, kỹ năng mới thông qua trải nghiệm, giáo dục, nghiên cứu. Quá trình học có thể trải qua các bước: quan sát, đặt giả thuyết, làm thí nghiệm, phân tích dữ liệu, học và báo cáo kết quả, chia sẻ kiến thức. 
Kiến thức, kỹ năng có thể được biểu diễn trong máy tính như là hàm ẩn tối ưu. 
Học máy là quá trình máy tính học qua các trải nghiệm. Máy tính học bằng cách tìm kiếm trong không gian hàm số/chương trình. Học máy giúp máy tính tự lập trình qua các trải nghiệm.
\section{Các thành phần cơ bản (TEFPA như trong bài giảng) cần được mô tả và cung cấp để máy tính tự học và giải quyết một tác vụ là gì?}
Các thành phần cơ bản (TEFPA như trong bài giảng) cần được mô tả và cung cấp để máy tính tự học và giải quyết một tác vụ là tìm giả thuyết/ hàm $\hat{f}$ $\in$ F có đô khái quát hóa cao nhất.
\section{Mô tả:} 
\textbf{Tác vụ (đầu vào và đầu ra là gì)\\
Kinh nghiệm/dữ liệu cần chuẩn bị(cần thu thập dữ liệu gì, cách thu thập ra sao, trong bao lâu, cần ai Giúp đỡ việc gì, mức độ khó khăn về thời gian công sức trang thiết bị v.v.) \\
cách thức đánh giá chất lượng đầu ra cho các ứng dụng sau:}


a) máy tính chuyển 1 tấm ảnh chất lượng kém (mờ, low-resolution) lên thành ảnh rõ nét (high-resolution)
Tác vụ: 
	Đầu vào: ảnh chất lượng kém (mờ, low-resolution)
	Đầu ra: ảnh rõ nét (high-resolution)
Kinh nghiệm/dữ liệu chuẩn bị: 
	Tập ảnh rõ nét và tập đã được làm mờ từ tập ảnh rõ nét
Cách thức đánh giá chất lượng đầu ra:
	Số lượng pixel chênh lệch giữa ảnh đầu vào và đầu ra là max

b) máy tính xử lý ảnh chụp X-quang và dự đoán bệnh
Tác vụ: 
	Đầu vào: ảnh chụp X-quang
	Đầu ra: bệnh
Kinh nghiệm/dữ liệu chuẩn bị: 
	Tập ảnh chụp X-quang và bệnh được chuẩn đoán tương ứng
	Khó khăn: dữ liệu bệnh nhân khó khai thác
Cách thức đánh giá chất lượng đầu ra:
	Bệnh được dự đoán giống với bệnh thật ứng với ảnh x-quang nhất
	
c) máy tính đọc một email của khách hàng và tự chuyển đến thư mục tương ứng như "cảm ơn", "khiếu nại", "hỏi thông tin", "xin việc", v.v.
Tác vụ: 
	Đầu vào: email của khách hàng
	Đầu ra: tên thư mục
Kinh nghiệm/dữ liệu chuẩn bị: 
	Tập email của khách hàng, nhóm thư mục
	Khó khăn: gom nhóm email theo thư mục
Cách thức đánh giá chất lượng đầu ra:
	Email có nội dung được phân loại đúng với tên thư mục nhất
	
\section{Ý nghĩa câu phát biểu sau: Máy tính "học" bằng cách tìm kiếm trong không gian các hàm số (chương trình máy tính)}
Ý nghĩa câu phát biểu sau: Máy tính "học" bằng cách tìm kiếm trong không gian các hàm số (chương trình máy tính là tìm hàm ẩn tối ưu cho một bài toán từ nhiệm vụ (task), tập dữ liệu (experience - dataset), hàm đánh giá (performance) và thuật toán để học (learning algorithm).

\section{Hai vấn đề chính trong không gian hàm mà ta cần đặc biệt chú ý để giúp máy tính tự tìm kiếm hàm có độ khái quát cao là gì?}
Hai vấn đề chính trong không gian hàm mà ta cần đặc biệt chú ý để giúp máy tính tự tìm kiếm hàm có độ khái quát cao là biểu diễn không gian hàm (representation) và tìm kiếm/huấn luyện (search/train/learn/optimize)/

\section{Chia sẻ với bạn bè về các điều mà học viên thấy lý thú qua bài giảng này}

Qua bài giảng này ta biết được các khái niệm cơ bản về trí tuệ nhân tạo và machine learning một cách dễ hiểu qua các ví dụ, các bài toán cụ thể được trình bày rất chi tiết.

\newpage
\bibliography{template}

\end{document}
